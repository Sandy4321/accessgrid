\documentclass{howto}

\title{Venue DataStore Client API}

% \release{0.5.1}

\author{Robert Olson}
\authoraddress{\email{olson@mcs.anl.gov}}

\usepackage[english]{babel}
\usepackage[T1]{fontenc}

\makeindex
\makemodindex

\begin{document}

\maketitle

\begin{abstract}
\noindent
The venue datastore provides a mechanism for the venue to store data
on behalf of the venue's users. The client API described here provides
easy access for the clients of a venue to access the data therein.
\end{abstract}

\tableofcontents

\section{Introduction \label{intro}}

%%%%%%%%%%%%%%%%%%%%%%%%%%%%%%%%%
%
% begin module docs
%
%%%%%%%%%%%%%%%%%%%%%%%%%%%%%%%%%

\section{\module{AccessGrid.VenueDataStoreClient} --- Client API to a
venue-based data store}

This module defines the interface to a Venue datastore, as available
beginning with AGTk version 2.0. 

\begin{funcdesc}{GetVenueDataStore}{venueURL}
Return a \class{DataStoreClient}  instance representing the default
datastore on the venue with url \var{venueURL}.
\end{funcdesc}

\section{\module{AccessGrid.DataStoreClient} --- Client API to the
Venue DataStore}

\declaremodule{}{AccessGrid.DataStoreClient}
\modulesynopsis{Client access to the Venue Datastore}


\begin{classdesc}{DataStoreClient}{uploadURL, datastoreURL}
A \class{DataStoreClient}  instance provides access to the datastore
represented by \var{uploadURL} and \var{datastoreURL}.

\end{classdesc}

\begin{classdesc}{DataStoreShell}{datastoreClient}

A \class{DataStoreShell} instance provides a command-line interface to
the files in a datastore.

\end{classdesc}



%%%%%%%%%%%%%%%%%%%%%%%%%%
%
% DataStoreClient methods.
%
%%%%%%%%%%%%%%%%%%%%%%%%%%


\subsection{DataStoreClient Objects}

The \class{DataStoreClient} class defines the following methods:

\begin{methoddesc}{LoadData}{}
Reload the client's cache of venue data information.
\end{methoddesc}

\begin{methoddesc}{QueryMatchingFiles}{pattern}
Return a list of filenames in the venue that match the unix-style file
glob \var{pattern}.

\end{methoddesc}

\begin{methoddesc}{QueryMatchingFilesMultiple}{patternList}
Return a list of filenames in the venue that match any of the unix-style file
globs in \var{patternList}.

\end{methoddesc}

\begin{methoddesc}{GetFileData}{filename}
Return the metadata in the cache for file \var{filename}. The metadata
is returned as a Python dictionary. Keys of
interest in this dictionary include 
\begin{description}
\item[name~] Name of the file
\item[size~] Size of the file, in bytes
\item[checksum~] MD5 checksum of the file
\item[uri~] DataStore URL for the file.
\end{description}

\end{methoddesc}

\begin{methoddesc}{Download}{filename, localFile}
Download \var{filename} to the local file \var{localFile}.

Raises a \var{FileNotFound} exception if \var{filename} does not exist
in the repository.

\end{methoddesc}

\begin{methoddesc}{Upload}{localFile}
Upload local file \var{localFile} to the datastore. The uploaded file will
be named with the basename of \var{localFile}. If that file already
exists, an exception will be raised.
\end{methoddesc}

\begin{methoddesc}{RemoveFile}{file}
Remove the file named \var{file} from the venue datastore.

\end{methoddesc}

\begin{methoddesc}{OpenFile}{file, mode}
Open a venue file named \var{file}. 

If \var{mode} is ``r'', the file
will be opened for reading. In this implementation, the file will be
downloaded and a filehandle to that local file returned. 

If \var{mode} is ``w'', the file will be opened for writing. In this
implemetnation, a file handle to a new local file will be
returned. When the file handle is closed, the file will be uploaded to
the venue.

\end{methoddesc}

\subsection{DataStoreShell Objects}

The \class{DataStoreShell} class defines the following methods:

\begin{methoddesc}{run}{}
Start the command processor.

\end{methoddesc}

%begin{latexonly}
\renewcommand{\indexname}{Module Index}
%end{latexonly}
\input{modDataStoreClientAPI.ind}              % Module Index

%begin{latexonly}
\renewcommand{\indexname}{Index}
%end{latexonly}
\documentclass{howto}

\title{Venue DataStore Client API}

% \release{0.5.1}

\author{Robert Olson}
\authoraddress{\email{olson@mcs.anl.gov}}

\usepackage[english]{babel}
\usepackage[T1]{fontenc}

\makeindex
\makemodindex

\begin{document}

\maketitle

\begin{abstract}
\noindent
The venue datastore provides a mechanism for the venue to store data
on behalf of the venue's users. The client API described here provides
easy access for the clients of a venue to access the data therein.
\end{abstract}

\tableofcontents

\section{Introduction \label{intro}}

%%%%%%%%%%%%%%%%%%%%%%%%%%%%%%%%%
%
% begin module docs
%
%%%%%%%%%%%%%%%%%%%%%%%%%%%%%%%%%

\section{\module{AccessGrid.VenueDataStoreClient} --- Client API to a
venue-based data store}

This module defines the interface to a Venue datastore, as available
beginning with AGTk version 2.0. 

\begin{funcdesc}{GetVenueDataStore}{venueURL}
Return a \class{DataStoreClient}  instance representing the default
datastore on the venue with url \var{venueURL}.
\end{funcdesc}

\section{\module{AccessGrid.DataStoreClient} --- Client API to the
Venue DataStore}

\declaremodule{}{AccessGrid.DataStoreClient}
\modulesynopsis{Client access to the Venue Datastore}


\begin{classdesc}{DataStoreClient}{uploadURL, datastoreURL}
A \class{DataStoreClient}  instance provides access to the datastore
represented by \var{uploadURL} and \var{datastoreURL}.

\end{classdesc}

\begin{classdesc}{DataStoreShell}{datastoreClient}

A \class{DataStoreShell} instance provides a command-line interface to
the files in a datastore.

\end{classdesc}



%%%%%%%%%%%%%%%%%%%%%%%%%%
%
% DataStoreClient methods.
%
%%%%%%%%%%%%%%%%%%%%%%%%%%


\subsection{DataStoreClient Objects}

The \class{DataStoreClient} class defines the following methods:

\begin{methoddesc}{LoadData}{}
Reload the client's cache of venue data information.
\end{methoddesc}

\begin{methoddesc}{QueryMatchingFiles}{pattern}
Return a list of filenames in the venue that match the unix-style file
glob \var{pattern}.

\end{methoddesc}

\begin{methoddesc}{QueryMatchingFilesMultiple}{patternList}
Return a list of filenames in the venue that match any of the unix-style file
globs in \var{patternList}.

\end{methoddesc}

\begin{methoddesc}{GetFileData}{filename}
Return the metadata in the cache for file \var{filename}. The metadata
is returned as a Python dictionary. Keys of
interest in this dictionary include 
\begin{description}
\item[name~] Name of the file
\item[size~] Size of the file, in bytes
\item[checksum~] MD5 checksum of the file
\item[uri~] DataStore URL for the file.
\end{description}

\end{methoddesc}

\begin{methoddesc}{Download}{filename, localFile}
Download \var{filename} to the local file \var{localFile}.

Raises a \var{FileNotFound} exception if \var{filename} does not exist
in the repository.

\end{methoddesc}

\begin{methoddesc}{Upload}{localFile}
Upload local file \var{localFile} to the datastore. The uploaded file will
be named with the basename of \var{localFile}. If that file already
exists, an exception will be raised.
\end{methoddesc}

\begin{methoddesc}{RemoveFile}{file}
Remove the file named \var{file} from the venue datastore.

\end{methoddesc}

\begin{methoddesc}{OpenFile}{file, mode}
Open a venue file named \var{file}. 

If \var{mode} is ``r'', the file
will be opened for reading. In this implementation, the file will be
downloaded and a filehandle to that local file returned. 

If \var{mode} is ``w'', the file will be opened for writing. In this
implemetnation, a file handle to a new local file will be
returned. When the file handle is closed, the file will be uploaded to
the venue.

\end{methoddesc}

\subsection{DataStoreShell Objects}

The \class{DataStoreShell} class defines the following methods:

\begin{methoddesc}{run}{}
Start the command processor.

\end{methoddesc}

%begin{latexonly}
\renewcommand{\indexname}{Module Index}
%end{latexonly}
\input{modDataStoreClientAPI.ind}              % Module Index

%begin{latexonly}
\renewcommand{\indexname}{Index}
%end{latexonly}
\documentclass{howto}

\title{Venue DataStore Client API}

% \release{0.5.1}

\author{Robert Olson}
\authoraddress{\email{olson@mcs.anl.gov}}

\usepackage[english]{babel}
\usepackage[T1]{fontenc}

\makeindex
\makemodindex

\begin{document}

\maketitle

\begin{abstract}
\noindent
The venue datastore provides a mechanism for the venue to store data
on behalf of the venue's users. The client API described here provides
easy access for the clients of a venue to access the data therein.
\end{abstract}

\tableofcontents

\section{Introduction \label{intro}}

%%%%%%%%%%%%%%%%%%%%%%%%%%%%%%%%%
%
% begin module docs
%
%%%%%%%%%%%%%%%%%%%%%%%%%%%%%%%%%

\section{\module{AccessGrid.VenueDataStoreClient} --- Client API to a
venue-based data store}

This module defines the interface to a Venue datastore, as available
beginning with AGTk version 2.0. 

\begin{funcdesc}{GetVenueDataStore}{venueURL}
Return a \class{DataStoreClient}  instance representing the default
datastore on the venue with url \var{venueURL}.
\end{funcdesc}

\section{\module{AccessGrid.DataStoreClient} --- Client API to the
Venue DataStore}

\declaremodule{}{AccessGrid.DataStoreClient}
\modulesynopsis{Client access to the Venue Datastore}


\begin{classdesc}{DataStoreClient}{uploadURL, datastoreURL}
A \class{DataStoreClient}  instance provides access to the datastore
represented by \var{uploadURL} and \var{datastoreURL}.

\end{classdesc}

\begin{classdesc}{DataStoreShell}{datastoreClient}

A \class{DataStoreShell} instance provides a command-line interface to
the files in a datastore.

\end{classdesc}



%%%%%%%%%%%%%%%%%%%%%%%%%%
%
% DataStoreClient methods.
%
%%%%%%%%%%%%%%%%%%%%%%%%%%


\subsection{DataStoreClient Objects}

The \class{DataStoreClient} class defines the following methods:

\begin{methoddesc}{LoadData}{}
Reload the client's cache of venue data information.
\end{methoddesc}

\begin{methoddesc}{QueryMatchingFiles}{pattern}
Return a list of filenames in the venue that match the unix-style file
glob \var{pattern}.

\end{methoddesc}

\begin{methoddesc}{QueryMatchingFilesMultiple}{patternList}
Return a list of filenames in the venue that match any of the unix-style file
globs in \var{patternList}.

\end{methoddesc}

\begin{methoddesc}{GetFileData}{filename}
Return the metadata in the cache for file \var{filename}. The metadata
is returned as a Python dictionary. Keys of
interest in this dictionary include 
\begin{description}
\item[name~] Name of the file
\item[size~] Size of the file, in bytes
\item[checksum~] MD5 checksum of the file
\item[uri~] DataStore URL for the file.
\end{description}

\end{methoddesc}

\begin{methoddesc}{Download}{filename, localFile}
Download \var{filename} to the local file \var{localFile}.

Raises a \var{FileNotFound} exception if \var{filename} does not exist
in the repository.

\end{methoddesc}

\begin{methoddesc}{Upload}{localFile}
Upload local file \var{localFile} to the datastore. The uploaded file will
be named with the basename of \var{localFile}. If that file already
exists, an exception will be raised.
\end{methoddesc}

\begin{methoddesc}{RemoveFile}{file}
Remove the file named \var{file} from the venue datastore.

\end{methoddesc}

\begin{methoddesc}{OpenFile}{file, mode}
Open a venue file named \var{file}. 

If \var{mode} is ``r'', the file
will be opened for reading. In this implementation, the file will be
downloaded and a filehandle to that local file returned. 

If \var{mode} is ``w'', the file will be opened for writing. In this
implemetnation, a file handle to a new local file will be
returned. When the file handle is closed, the file will be uploaded to
the venue.

\end{methoddesc}

\subsection{DataStoreShell Objects}

The \class{DataStoreShell} class defines the following methods:

\begin{methoddesc}{run}{}
Start the command processor.

\end{methoddesc}

%begin{latexonly}
\renewcommand{\indexname}{Module Index}
%end{latexonly}
\input{modDataStoreClientAPI.ind}              % Module Index

%begin{latexonly}
\renewcommand{\indexname}{Index}
%end{latexonly}
\documentclass{howto}

\title{Venue DataStore Client API}

% \release{0.5.1}

\author{Robert Olson}
\authoraddress{\email{olson@mcs.anl.gov}}

\usepackage[english]{babel}
\usepackage[T1]{fontenc}

\makeindex
\makemodindex

\begin{document}

\maketitle

\begin{abstract}
\noindent
The venue datastore provides a mechanism for the venue to store data
on behalf of the venue's users. The client API described here provides
easy access for the clients of a venue to access the data therein.
\end{abstract}

\tableofcontents

\section{Introduction \label{intro}}

%%%%%%%%%%%%%%%%%%%%%%%%%%%%%%%%%
%
% begin module docs
%
%%%%%%%%%%%%%%%%%%%%%%%%%%%%%%%%%

\section{\module{AccessGrid.VenueDataStoreClient} --- Client API to a
venue-based data store}

This module defines the interface to a Venue datastore, as available
beginning with AGTk version 2.0. 

\begin{funcdesc}{GetVenueDataStore}{venueURL}
Return a \class{DataStoreClient}  instance representing the default
datastore on the venue with url \var{venueURL}.
\end{funcdesc}

\section{\module{AccessGrid.DataStoreClient} --- Client API to the
Venue DataStore}

\declaremodule{}{AccessGrid.DataStoreClient}
\modulesynopsis{Client access to the Venue Datastore}


\begin{classdesc}{DataStoreClient}{uploadURL, datastoreURL}
A \class{DataStoreClient}  instance provides access to the datastore
represented by \var{uploadURL} and \var{datastoreURL}.

\end{classdesc}

\begin{classdesc}{DataStoreShell}{datastoreClient}

A \class{DataStoreShell} instance provides a command-line interface to
the files in a datastore.

\end{classdesc}



%%%%%%%%%%%%%%%%%%%%%%%%%%
%
% DataStoreClient methods.
%
%%%%%%%%%%%%%%%%%%%%%%%%%%


\subsection{DataStoreClient Objects}

The \class{DataStoreClient} class defines the following methods:

\begin{methoddesc}{LoadData}{}
Reload the client's cache of venue data information.
\end{methoddesc}

\begin{methoddesc}{QueryMatchingFiles}{pattern}
Return a list of filenames in the venue that match the unix-style file
glob \var{pattern}.

\end{methoddesc}

\begin{methoddesc}{QueryMatchingFilesMultiple}{patternList}
Return a list of filenames in the venue that match any of the unix-style file
globs in \var{patternList}.

\end{methoddesc}

\begin{methoddesc}{GetFileData}{filename}
Return the metadata in the cache for file \var{filename}. The metadata
is returned as a Python dictionary. Keys of
interest in this dictionary include 
\begin{description}
\item[name~] Name of the file
\item[size~] Size of the file, in bytes
\item[checksum~] MD5 checksum of the file
\item[uri~] DataStore URL for the file.
\end{description}

\end{methoddesc}

\begin{methoddesc}{Download}{filename, localFile}
Download \var{filename} to the local file \var{localFile}.

Raises a \var{FileNotFound} exception if \var{filename} does not exist
in the repository.

\end{methoddesc}

\begin{methoddesc}{Upload}{localFile}
Upload local file \var{localFile} to the datastore. The uploaded file will
be named with the basename of \var{localFile}. If that file already
exists, an exception will be raised.
\end{methoddesc}

\begin{methoddesc}{RemoveFile}{file}
Remove the file named \var{file} from the venue datastore.

\end{methoddesc}

\begin{methoddesc}{OpenFile}{file, mode}
Open a venue file named \var{file}. 

If \var{mode} is ``r'', the file
will be opened for reading. In this implementation, the file will be
downloaded and a filehandle to that local file returned. 

If \var{mode} is ``w'', the file will be opened for writing. In this
implemetnation, a file handle to a new local file will be
returned. When the file handle is closed, the file will be uploaded to
the venue.

\end{methoddesc}

\subsection{DataStoreShell Objects}

The \class{DataStoreShell} class defines the following methods:

\begin{methoddesc}{run}{}
Start the command processor.

\end{methoddesc}

%begin{latexonly}
\renewcommand{\indexname}{Module Index}
%end{latexonly}
\input{modDataStoreClientAPI.ind}              % Module Index

%begin{latexonly}
\renewcommand{\indexname}{Index}
%end{latexonly}
\input{DataStoreClientAPI.ind}                 % Index

\end{document}

                 % Index

\end{document}

                 % Index

\end{document}

                 % Index

\end{document}

